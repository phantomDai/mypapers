

\documentclass[10pt,journal,cspaper,compsoc,onecolumn]{IEEEtran}

\usepackage{algorithm}
\usepackage{algorithmic}
\usepackage{amssymb}
\usepackage{color}
\usepackage{graphicx}
\usepackage{subfigure}
\usepackage{amsmath}
\usepackage{setspace}
\allowdisplaybreaks[4]
\hyphenation{op-tical net-works semi-conduc-tor}


\begin{document}
\begin{spacing}{1.3}

\large{\textbf{Appendix A: The proofs of Lemma 1 and Theorem 1}}

Derivation of the relationship between $p_i^{n + 1}$ and $p_i^{n}$ is as follows:
\begin{equation}
    \label{eq:7}
    \begin{split}
    p_i^{n + 1} = &p_i^n\theta_i(p_i^n + \varepsilon) + p_i^n(1-\theta_i)(p_i^n - \varepsilon)\\
    &+ \sum_{j \neq i}p_j^n\theta_j(p_i^n - \displaystyle \frac{\varepsilon}{m- 1})\\
    &+ \sum_{j \neq i}p_j^n(1 - \theta_j)(p_i^n + \displaystyle\frac{\varepsilon}{m - 1})\\
    =&(p_i^n)^2\theta_i + p_i^n\theta_i\varepsilon + (p_i^n)^2 - p_i^n\varepsilon - (p_i^n)^2\theta_i + p_i^n\theta_i\varepsilon\\
    & + (p_i^n - \displaystyle\frac{\varepsilon}{m - 1})\sum_{j \neq i}p_j^n\theta_j + (p_i^n + \displaystyle\frac{\varepsilon}{m - 1})\sum_{j \neq i}p_j^n\\
    &- (p_i^n + \displaystyle\frac{\varepsilon}{m - 1})\sum_{j \neq i}p_j^n\theta_j\\
    =&(p_i^n)^2 + 2p_i^n\theta_i\varepsilon - p_i^n\varepsilon + (p_i^n - \displaystyle\frac{\varepsilon}{m - 1} - p_i^n \\
    &- \displaystyle\frac{\varepsilon}{m - 1})\sum_{j \neq i}p_j^n\theta_j + (p_i^n + \displaystyle\frac{\varepsilon}{m - 1})(1 - p_i^n)\\
    =&p_i^n + (p_i^n)^2 - (p_i^n)^2 + 2p_i^n\theta_i\varepsilon - p_i^n\varepsilon + \displaystyle\frac{\varepsilon}{m - 1} - \\
    &\displaystyle\frac{\varepsilon}{m - 1}p_i^n - \displaystyle\frac{2\varepsilon}{m - 1}\sum_{j \neq i}p_j^n\theta_j\\
    =&p_i^n + \displaystyle\frac{\varepsilon}{m - 1}(2p_i^n\theta_im - p_i^nm - 2p_i^n\theta_i + 1 )\\
    &- \displaystyle\frac{2\varepsilon}{m-1}\sum_{j \neq i}p_j^n\theta_j\\
    =&p_i^n + Y_i^n,
    \end{split}
\end{equation}
where
\begin{equation}
    \begin{split}
    \label{eq:yi}
    Y_i^n = &\displaystyle\frac{\varepsilon}{m - 1}(2p_i^n\theta_im - p_i^nm - 2p_i^n\theta_i + 1 )\\
             &- \displaystyle\frac{2\varepsilon}{m-1}\sum_{j \neq i}p_j^n\theta_j.
    \end{split}
 \end{equation}
 
Derivation of the relationship between $Y_M^{n}$ and $Y_i^{n}$ is as follows:
 \begin{align}
    \label{eq:26}
    Y_M^n - Y_i^n = &\displaystyle\frac{\varepsilon}{m - 1}(2p_M^n\theta_Mm - p_M^nm - 2p_M^n\theta_M + 1)\nonumber \\
    & - \displaystyle\frac{2\varepsilon}{m - 1}\sum_{j \neq M}p_j^n\theta_j - \displaystyle\frac{\varepsilon}{m - 1}(2p_i^n\theta_im - p_i^nm\nonumber\\
    & - 2p_i^n\theta_i + 1) + \displaystyle\frac{2\varepsilon}{m - 1}\sum_{j \neq i}p_j^n\theta_j \nonumber\\
    =&\displaystyle\frac{\varepsilon}{m - 1}(2m(p_M^n\theta_M - p_i^n\theta_i) - m(p_M^n - p_i^n) -\nonumber\\
    & 2(p_M^n\theta_M - p_i^n\theta_i)) - \sum_{j \neq M}p_j^n\theta_j + \sum_{j \neq i}p_j^n\theta_j     \nonumber\\ 
    =& \displaystyle\frac{2\varepsilon}{m - 1}(m(p_M^n\theta_M - p_i^n\theta_i) - \displaystyle\frac{m(p_M^n - p_i^n)}{2} - \nonumber\\
    &(p_M^n\theta_M - p_i^n\theta_i)) + \displaystyle\frac{2\varepsilon}{m-1}(p_M^n\theta_M - p_i^n\theta_i)\nonumber\\
    =&\displaystyle\frac{2\varepsilon}{m - 1}(m(p_M^n\theta_M - p_i^n\theta_i) - \displaystyle\frac{m(p_M^n - p_i^n)}{2}).
\end{align}

\newtheorem{lem_appendix}{Lemma}
\label{Lemma_appdendix}
\begin{lem_appendix}
  If $p_i^{n} - p_M^{n} > 2(p_i^{n}\theta_i - p_M^{n}\theta_M)$, then $p_M^{n + 1} > p_M^{n}$.
\end{lem_appendix}
\begin{IEEEproof}
The condition $p_i^{n} - p_M^{n} > 2(p_i^{n}\theta_i - p_M^{n}\theta_M)$ can be equivalently expressed as:
\begin{equation}
  \label{lemma:proof1}
\displaystyle\frac{p_M^n - p_i^n}{2} < p_M^{n}\theta_M - p_i^{n}\theta_i.
\end{equation}

From Formula~\ref{lemma:proof1},  $(p_M^{n}\theta_M - p_i^{n}\theta_i) - \displaystyle\frac{p_M^n - p_i^n}{2} > 0$, and because $0 < \varepsilon < 1$, and $m > 1$,  therefore:
\begin{equation}
  \label{lemma:proof2}
\displaystyle\frac{2m\varepsilon}{m - 1}((p_M^{n}\theta_M - p_i^{n}\theta_i) - \displaystyle\frac{p_M^n - p_i^n}{2}) > 0.
\end{equation}

Furthermore:
\begin{equation}
  \label{lemma:proof3}
\displaystyle\frac{2\varepsilon}{m - 1}(m(p_M^{n}\theta_M - p_i^{n}\theta_i) - \displaystyle\frac{m(p_M^n - p_i^n)}{2}) > 0.
\end{equation}

According to Formulas \ref{lemma:proof3} and \ref{eq:26},
if $p_i^{n} - p_M^{n} > 2(p_i{n}\theta_i - p_M^{n}\theta_M)$, then
$Y_M^n - Y_i^n > 0$.

Also, because
$\sum_{i=1}^mp_i^{n + 1} = 1$, and
$\sum_{i=1}^mp_i^{n} = 1$, therefore
$Y_M^n >0$, and thus
$\sum_{i=1}^mY_i^n = 0$.

According to Formula~\ref{eq:7},
$p_M^{n + 1} = p_M^n + Y_M^n$.
Because $Y_M^n >0$, therefore
$p_M^{n + 1} > p_M^n$.
\end{IEEEproof}

\newtheorem{my}{Theorem}
\label{theorem}
\begin{my}
  For failure rate $\theta_{min} = min\{\theta_1, \ldots, \theta_m\}$, $\theta_M > \theta_{min}$, if $0 < \theta_{min} < \frac{1}{2}$, the following condition is sufficient to guarantee that $p_M^{n + 1} > p_M^{n}$:
  \begin{equation}
\label{equa:results}
  \displaystyle\frac{2m\theta_{min}^2}{1-2\theta_{min}} < \varepsilon < \displaystyle\frac{(m-1)m\theta_{min}}{2(m + 1)}.
\end{equation}
\end{my}

\begin{IEEEproof}
  In order to guarantee $p_M^{n + 1} > p_M^{n}$, we consider the following three situations (where $i \in \{1, 2, \ldots, m\}$ and $i \ne M$).

  \textbf{Situation 1 ($p_i^n = p_M^n$):}
Because $\theta_i < \theta_M$, therefore
$(p_i^n\theta_i - p_M^n\theta_M) < 0$.

Therefore, $(p_i^n - p_M^n) > 2(p_i^n\theta_i - p_M^n\theta_M)$.

According to Lemma 1, we have $p_M^{n + 1} > p_M^{n}$.

\textbf{Situation 2 ($p_i^n > p_M^n$):} According to Formula \ref{equa:results}, we have the following:
$$\varepsilon > \displaystyle\frac{2m\theta_{min}^2}{1-2\theta_{min}}.$$

Because
$$\displaystyle\frac{2m\theta_{min}^2}{1-2\theta_{min}} = \displaystyle\frac{\theta_{min}}{1/2m\theta_{min} - 1/m},$$
we have the following:
$$\varepsilon > \displaystyle\frac{\theta_{min}}{1/2m\theta_{min} - 1/m}.$$

Because $\theta_{min} < 1/2$, therefore
$1/2m\theta_{min} - 1/m > 0$ and
$\varepsilon(1/2m\theta - 1/m) > \theta_{min}$, which gives
$\varepsilon/2m\theta_{min} > \theta_{min} + \varepsilon/m$.

Because $\varepsilon > 0$, and $m > 1$, therefore
 $$\displaystyle\frac{1}{2\theta_{min}} > \displaystyle\frac{(\theta_{min} + \varepsilon/m)}{(\varepsilon/m)}.$$

$(1/2\theta_{min})(p_i^n - p_M^n) > (p_i^n - p_M^n)(\theta_{min} + \varepsilon/m)/(\varepsilon/m)$ as
$p_i^n > p_M^n$, and
$$p_i^n -p_M^n > 2\theta_{min}(p_i^n -p_M^n)\displaystyle\frac{\theta_{min} + \varepsilon/m}{\varepsilon/m}.$$

Because $(\theta_{min} + \varepsilon/m)/(\varepsilon/m) > 1$, therefore
$$2\theta_{min}(p_i^n -p_M^n)\displaystyle\frac{\theta_{min} + \varepsilon/m}{\varepsilon/m} > 2\theta_{min}(p_i^n -p_M^n).$$

Because $\theta_{min} < \theta_M$, therefore
$$2\theta_{min}(p_i^n - p_M^n) > 2(p_i^n\theta_{min} - p_M^n\theta_M).$$

Thus,
$$p_i^n -p_M^n > 2(p_i^n\theta_{min} - p_M^n\theta_M).$$

According to Lemma 1, we have $p_M^{n + 1} > p_M^{n}$.

\textbf{Situation 3 ($p_i^n < p_M^n$):}
For this proof, we make the assumption that $\frac{1}{2} < \theta_M < 1$.

Because we have
$$\varepsilon < \displaystyle\frac{(m - 1)m\theta_{min}}{2(m + 1)}$$
and
$$\displaystyle\frac{(m - 1)m\theta_{min}}{2(m + 1)} = \displaystyle\frac{2m -(m + 1)}{2(m + 1)}m\theta_{min},$$
thus
$$\varepsilon < (\displaystyle\frac{m}{m +1} - \displaystyle\frac{1}{2})m\theta_{min}.$$

Obviously, $\varepsilon/m < (m/(m + 1) - 1/2)\theta_{min}$ as $m > 1$.

Furthermore, we have
$$-\displaystyle\frac{\varepsilon}{m} > (\displaystyle\frac{1}{2} - \displaystyle\frac{m}{m + 1})\theta_{min}$$
and
$$\displaystyle\frac{m\theta_{min}}{m + 1} - \displaystyle\frac{\varepsilon}{m} + \displaystyle\frac{2\varepsilon}{m} > \displaystyle\frac{\theta_{min}}{2} + \displaystyle\frac{2\varepsilon}{m}$$
which means that
$$\displaystyle\frac{m\theta_{min}}{m + 1} + \displaystyle\frac{\varepsilon}{m}  > \displaystyle\frac{1}{2}(\theta_{min} + \displaystyle\frac{4\varepsilon}{m}).$$

It follows that
$$(m\theta_{min}/(m + 1) + \varepsilon/m)/(4\varepsilon/m + \theta_{min}) > 1/2$$ for any $m > 1, \varepsilon > 0$, and $0 < \theta_{min} < 1$.

Because $\frac{1}{2} < \theta_M < 1$, therefore
$(m\theta_{min}/(m + 1) + \varepsilon/m)/(4\varepsilon/m + \theta_{min}) > 1/2\theta_M$.

Thus, we have
$$2(p_M^n -p_i^n)\theta_M\displaystyle\frac{\displaystyle\frac{\varepsilon}{m} +\displaystyle\frac{m\theta_{min}}{m + 1}}{\displaystyle\frac{4\varepsilon}{m} + \theta_{min}} > p_M^n -p_i^n$$
as $p_M^n > p_i^n$.

Because $\varepsilon/m < 4\varepsilon/m$, and $m\theta_{min}/(m + 1) < \theta_{min}$, therefore
$$ \displaystyle\frac{\displaystyle\frac{\varepsilon}{m} +\displaystyle\frac{m\theta_{min}}{m + 1}}{\displaystyle\frac{4\varepsilon}{m} + \theta_{min}} < 1$$
and
$$ 2(p_M^n -p_i^n)\theta_M > 2(p_M^n -p_i^n)\theta_M\displaystyle\frac{\displaystyle\frac{\varepsilon}{m} +\displaystyle\frac{m\theta_{min}}{m + 1}}{\displaystyle\frac{4\varepsilon}{m} + \theta_{min}}$$

Hence we have
$$2(p_M^n -p_i^n)\theta_M > p_M^n -p_i^n,$$
which can be equivalently expressed as
$$p_i^n -p_M^n > 2(p_i^n - p_M^n)\theta_M.$$

Because $\theta_{min} < \theta_M$, therefore
$2(p_i^n - p_M^n)\theta_M > 2(p_i^n\theta_{min} - p_M^n\theta_M)$, and thus
$$p_i^n -p_M^n > 2(p_i^n\theta_{min} - p_M^n\theta_M).$$

According to Lemma 1, we have $p_M^{n + 1} > p_M^{n}$.
\end{IEEEproof}

\end{spacing}
\end{document}


